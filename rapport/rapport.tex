\documentclass[11pt,french,a4paper]{article}

\usepackage[utf8]{inputenc}
\usepackage[T1]{fontenc}

\usepackage[french]{babel}
\usepackage[a4paper,fancysections]{polytechnique}
\usepackage[hidelinks]{hyperref}
\usepackage{url}
\usepackage{shortvrb}
\MakeShortVerb{\|}

\title{Rapport de projet INF441}
\subtitle{Les liens dansants}
\author{Antoine \bsc{Berthier}\\ Denis \bsc{Merigoux}}
\date{Mardi 2 juin 2015}

\begin{document}

\maketitle

\section{Organisation du travail}

\paragraph{GitHub} Afin de coordonner nos efforts et assurer un suivi optimal de notre projet, nous avons utilisé un contrôle de version Git collaboratif. Notre projet est donc également disponible à l'adresse \url{https://github.com/denismerigoux/ProjetINF441}.

\paragraph{Contributions} Nous nous sommes répartis le travail comme il suit :
\begin{itemize}
	\item Denis a établi les modèles de données, écrit l'algorithme et s'est occupé des aspects plus théoriques du problème ;
	\item Antoine a réalisé les méthodes d'entrée/sortie et de création d'un objet utilisable par l'algorithme à partir de l'entrée imposée.
\end{itemize}

\section{Structure du code}

Toutes les fichiers source se trouvent dans le dossier |src| conformément à l'énoncé. On distingue trois catégories dans ces fichiers.

\paragraph{Algorithme} D'abord les fichiers liés à l'algorithme DLX et à la structure de données permettant de le réaliser :
\begin{itemize}
	\item |LinkObject|, classe abstraite modélisant implémentées par tous les objects ci-dessous ;
	\item |DataObject| qui modélise les 1 dans les matrices EMC ;
	\item |ColmunObject| qui modélise les en-têtes de colonnes ;
	\item |RootObject| pour l'objet \emph{root} décrit dans l'article ;
	\item |LinkMatrix| qui encapsule tous ces objets et contient la méthode exécutant l'algorithme DLX.
\end{itemize}

\paragraph{Parsing} Les fichiers suivants servent à transformer un problème EMC ou de pavage tel qu'il est donné dans l'énoncé en un object |LinkMatrix| utilisable par l'algorithme :
\begin{itemize}
	\item |EMCParsing| pour l'argument |-emc| de |Main| ;
	\item |PavageParsing| pour l'argument |-pavage| de |Main| ;
	\item |Piece| qui modélise les pièces à paver. 
\end{itemize}

\paragraph{Entrée-sortie, test, debug} Ces fichiers 





\end{document}